\documentclass[fleqn]{article}
\usepackage[a4paper,bindingoffset=0.2in,%
            left=1in,right=1in,top=1in,bottom=1in,%
            footskip=.25in]{geometry}
\usepackage{graphicx}
\usepackage{amsmath}

\begin{document}

\title{Tic-Tac-Toe}
\author{Szalontai Jordán}

\maketitle

\begin{abstract}
Háromszemélyes 4x4-es tic-tac-toe játék állapottér-reprezentációja
\end{abstract}

\section{Játékállások reprezentációja}
Legyen $J = \{1, 2, 3\}$ a játékosok szimbólumai, továbbá $H = J \cup \{ 0 \}$ és definiáljuk az\\$S: H^{4x4} \rightarrow \{H^{3x3},H^{3x3},H^{3x3},H^{3x3}\}$ függvény a következő képpen:
\\\\
$S\Bigg(\begin{bmatrix}
	h_{11} & h_{12} & h_{13} & h_{14} \\
	h_{21} & h_{22} & h_{23} & h_{24} \\
	h_{31} & h_{32} & h_{33} & h_{34} \\
	h_{41} & h_{42} & h_{43} & h_{44}
\end{bmatrix}\Bigg)$ = 
\Bigg\{$\begin{bmatrix}
	h_{11} & h_{12} & h_{13}\\
	h_{21} & h_{22} & h_{23}\\
	h_{31} & h_{32} & h_{33}\\
\end{bmatrix}$,
$\begin{bmatrix}
	h_{12} & h_{13} & h_{14} \\
	h_{22} & h_{23} & h_{24} \\
	h_{32} & h_{33} & h_{34} \\
\end{bmatrix}$,
$\begin{bmatrix}
	h_{21} & h_{22} & h_{23} \\
	h_{31} & h_{32} & h_{33} \\
	h_{41} & h_{42} & h_{43}
\end{bmatrix}$,
$\begin{bmatrix}
	h_{22} & h_{23} & h_{24} \\
	h_{32} & h_{33} & h_{34} \\
	h_{42} & h_{43} & h_{44}
\end{bmatrix}$\Bigg\}.
\\\\
\\
A játékállások reprezentációja egy $\langle\mathcal{A},  a_0, \mathcal{V}, \mathcal{L}\rangle$ négyes, ahol\\\\
$\mathcal{A} = H^{4x4}$, \\\\
$a_0 = \begin{bmatrix} 0 & 0 & 0 & 0 \\ 0 & 0 & 0 & 0 \\0 & 0 & 0 & 0 \\0 & 0 & 0 & 0 \end{bmatrix} \in \mathcal{A}$, \\\\
$\mathcal{L} = \{ l_{ijk} \mid i = 1, ..., 4 \wedge j = 1, ..., 4 \wedge k \in J \}$, \\\\
$\mathcal{V} = \{ v \in \mathcal{A} \mid \exists{s}(s \in S(v) \wedge (ROW(s) \vee COL(s) \vee DIAG(s))) \vee DRAW(v) \}$\\\\
rendre a játékállások halmaza, a kezdőállás, a lépések halmaza, valamint a végállások halmaza.
\\\\
A $ROW(s)$, $COL(s)$, $DIAG(s)$ formulák rendre megadják, hogy az adott $s \in H^{3x3}$ mátrixban egy sorban, egy oszlopban vagy valamelyik átlóban ugyanolyan nemnulla értékek szerepelnek-e, valamint a $DRAW(v)$ függvény megadja, az adott $v \in \mathcal{A}$ állás döntetlen-e. Ezek definíciói:\\\\
$ROW(s) \rightleftharpoons \forall{i}(s_{i1} \neq 0 \wedge s_{i1} = s_{i2} \wedge s_{i2} = s_{i3})$,\\
$COL(s) \rightleftharpoons \forall{j}(s_{1j} \neq 0 \wedge s_{1j} = s_{2j} \wedge s_{2j} = s_{3j}) \hfill  i = 1, ..., 3 \wedge j = 1, ..., 3$,\\
$DIAG(s) \rightleftharpoons s_{22} \neq 0 \wedge ((s_{11} = s_{22} \wedge s_{22} = s_{33}) \vee (s_{13} = s_{22} \wedge s_{31} = s_{33})) $\\
$DRAW(v) = \forall{x}(\forall{y}(v_{xy} \neq 0)) \hfill x = 1, ..., 4 \wedge y = 1, ..., 4$\\
\\
Továbbá $l_{ijk} \in \mathcal{L}$ esetén $dom(l_{ijk}) = \{ a \in \mathcal{A} \mid a_{ij} = 0\}$ és\\\\
$l_{ijk}\Bigg(\begin{bmatrix}
	a_{11} & a_{12} & a_{13} & a_{14} \\
	a_{21} & a_{22} & a_{23} & a_{24} \\
	a_{31} & a_{32} & a_{33} & a_{34} \\
	a_{41} & a_{42} & a_{43} & a_{44}
\end{bmatrix}\Bigg) = \begin{bmatrix}
	b_{11} & b_{12} & b_{13} & b_{14} \\
	b_{21} & b_{22} & b_{23} & b_{24} \\
	b_{31} & b_{32} & b_{33} & b_{34} \\
	b_{41} & b_{42} & b_{43} & b_{44}
\end{bmatrix},$\\\\\\
ahol
$
    b_{pq}=
\begin{cases}
    k,& \text{ha } p = i \wedge q = j \\
    a_{pq},              & \text{egyébként}
\end{cases}
\hfill  p = 1, ..., 4 \wedge q = 1, ..., 4\\
$


\section{Állapottér-reprezentáció}

A játék állapottér-reprezentációja egy $\langle\mathcal{A'},  a'_0, \mathcal{V'}, \mathcal{O}\rangle$ négyes, ahol\\\\
$\mathcal{A'} = \mathcal{A} x J$,\\\\
$a'_{0} = (a_{0}, 1)$,\\\\
$\mathcal{V'} = \{ (a, r) \in \mathcal{A'} \mid a \in \mathcal{V} \},$\\\\
$\mathcal{O} = \{ o_{ij} \mid i = 1, ..., 4 \wedge j = 1, ..., 4 \},$\\\\
továbbá $o_{ij} \in \mathcal{O}$, $a' \in \mathcal{A'}$ és $k \in J$ esetén $dom(o_{ij}) = \{ (a, r) \in \mathcal{A'} \mid a \in dom(l_{ijr})\}$ és\\ $o_{ij}((a', k)) = (l_{ijk}(a), NEXT(k))$.
\\\\
A $NEXT: J \rightarrow J$ függvény megadja a soron következő játékos szimbólumát.


\end{document}